\opt{text}{\skbheading{Loading Acronyms and Bibliographic Information}}

\DescribeMacro{\skbacronyms}
\DescribeMacro{\skbbibtex}
These two macros can be used to load the acronym database (\cmd{\skbacronyms}) and the references
(\cmd{\skbbibtex}). Both macros behave identical: they use \cmd{\InputIfFileExists} to check whether
the acronym or bibtex file exists. If so, they simply input the file. If not, they use \cmd{\PackageError} to 
throw an error with a help message, showing the requested database file to input.
One should use \cmd{\skbacronyms} at the place in the document were the list of acronyms should 
be printed and \cmd{\skbbibtex} at the beginning of the document to load the bibliographic information.
