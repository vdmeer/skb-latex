\opt{text}{\skbheading{Path and File Names}}

\DescribeMacro{\skbfileroot}
\DescribeMacro{\skbpathroot}
\DescribeMacro{\skbfileacr}
\DescribeMacro{\skbfilebib}
\DescribeMacro{\skbpathbib}
\DescribeMacro{\skbfilerep}
\DescribeMacro{\skbfilepub}
\DescribeMacro{\skbfilefig}
\DescribeMacro{\skbfilesli}
The \skbacft{vdm:skb} provides a number of macros to directly create path and file names. Most of 
these macros are actually used within the \skbacft{vdm:skb}, but they might also be useful for 
users to access files without using the provided specialised macros (such as \cmd{\skbinput}.
The following macros are provided:
\begin{skbnotelist}
  \item \cmd{\skbpathroot} -- returns the set root path of the \skbacft{vdm:skb}.
  \item \cmd{\skbfileroot} -- returns the set root path and adds \skbem[code]{/#1}, i.e. the directory separator and the argument provided.
  \item \cmd{\skbfileacr} -- returns the path (including root) and file name for the acronym database.
  \item \cmd{\skbfilebib} -- returns the path (including root) and file name for the file that loads the reference database (\BibTeX).
  \item \cmd{\skbpathbib} -- returns the path (including root) to the reference database.
  \item \cmd{\skbfilerep} -- returns the path to the repository and adds \skbem[code]{/#1}, i.e. the directory separator and the argument provided.
  \item \cmd{\skbfilepub} -- returns the path to the folder with the published documents and adds \skbem[code]{/#1}, i.e. the directory separator and the argument provided.
  \item \cmd{\skbfilefig} -- returns the path to the figure folder and adds \skbem[code]{/#1}, i.e. the directory separator and the argument provided.
  \item \cmd{\skbfilesli} -- returns the path to the slide folder and adds \skbem[code]{/#1}, i.e. the directory separator and the argument provided.
\end{skbnotelist}
