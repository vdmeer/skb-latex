\opt{text}{\skbheading{Slides}}

\DescribeMacro{\skbslide}
This macro helps to create lecture notes (handouts) using \skbacft{org:iso:pdf} slides and \LaTeX~notes
without using the beamer package. The reason for adding this to the \skbacft{vdm:skb} was to integrate
slides from sources outside the \LaTeX~universe (i.e. Microsoft Powerpoint). Some of my 
presentations are done using Powerpoint, but for handouts I do prefer using \LaTeX~thus
benefiting from many of the automated features it provides (references, acronyms). As a nice
side effect, the output generated is scalable (assuming that the \skbacft{org:iso:pdf} sources of the slides
contain scalable images rather than bitmaps, which can be easily realised using for instance
Inkscape's EMF export within Microsoft Powerpoint slides).

The macro \cmd{\skbslide}
provides all means to include \skbacft{org:iso:pdf} slides with or without annotations, annotations only and
it can load the annotations using different mechanisms.
The macro offers two options to set the input
path for the slides and the annotations: \skbem[code]{slidefrom} and \skbem[code]{notefrom}. 
If \skbem[code]{slidefrom} is used, then the slide (\skbacft{org:iso:pdf}) file will be loaded from the requested path (\skbem[code]{sli}, \skbem[code]{rep} or \skbem[code]{pub}).
If \skbem[code]{notefrom} is used, then the annotation (\TeX~) file will be loaded from the requested path (\skbem[code]{sli}, \skbem[code]{rep} or \skbem[code]{pub}).
The default path for slides and annotations is the path for slides.

The third option \skbem[code]{annotate} requests to load annotations. If not used, no annotations will be loaded. It can be used in 
combination with the two arguments to automated loading annotations.

The two arguments of this macro define the files for the slide and the annotation. They can be used as followes:
\begin{skbnotelist}
  \item Argument 1 is the slide to be loaded. If a name if given, we load the \skbacft{org:iso:pdf} using \cmd{\inputgraphics} with width being \cmd{\textwidth}. If no 
        name is given, no slide will be loaded.
  \item Argument 2 is the file with the annotations in combination with the option \skbem[code]{annotate}. If this option is not used then
        no annotations will be loaded. If the option is used and no name is given, then the annotation is loaded from a file with the 
        same name as the slide plus the extension \skbem[code]{.tex}. If this option is used and a name is given then this file will be loaded.
\end{skbnotelist}

\opt{text}{\skbinput[from=rep]{manual/slides2}}