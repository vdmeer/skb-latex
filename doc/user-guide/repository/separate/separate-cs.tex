\opt{text}{\skbheading{Separate Content from Structure: the Repository Folder}}

Now, separating the structure from the content first. The structure of a 
document (we stay with classic text documents like articles, books, etc. 
for a while) is words in sentences in paragraphs in (sub-) sections in 
chapters (if its a book, of not only sections)\footnote{One very 
meticulous person might add 'characters' and mention that there are more 
ways to think about a document's structure. But that person is not me. The 
structure I used fits the purpose (as of now), if it doesn't anymore I 
will look further.}. We collect sentences and paragraphs but separate them 
from headings. \LaTeX~is doing that already with the macros for chapters and 
sections. We go one step further and provide a generic way to identify a 
heading with the \skbacft{A3DS:SKB} macro \cmd{\setheading}. This allows to select the appropriate
\LaTeX~heading level at a later stage having the context of that later stage in mind
(i.e. it might be a section in an article but a chapter in a book). Now we create 
a structure for the resulting files in our repository, adding meaningful names to 
the directories and files. Obviously the names of these folders should provide some
idea about the general characterisation of the files they contain. Example? Well, if you collect 
information from \acs{SDO} the top folder 
could be named \skbem[code]{sdo} and the sub-folders using the 
respective \ac{SDO} acronyms, such as \skbem[code]{omg} for the \skbacft{organisation:OMG} and 
\skbem[code]{ieee} for the \skbacft{organisation:IEEE} and \skbem[code]{ietf} for the
\skbacft{organisation:IETF}. We put all this in a folder named repository, making it explicit
that here is were we find all our normative content.
\opt{text}{The following example}\opt{note}{This slide} shows how that looks like.

\opt{text}{
  \begin{longtable}{|>{}p{0.945\textwidth} >{}p{1pt}|}
    \hline
    \rowcolor[gray]{.9}
      \skbinput[from=fig]{dirtree/repository}
    & \\
    \hline
  \end{longtable}
  \addtocounter{table}{-1}
}

The result: we have a structure of files, containing our 'stuff', 
integrated in a structure of folders that allows us to find it (the better 
this structure the more helpful it is, and remember this is a 'personal' 
repository, so your personal flavour is king).