\opt{text}{\skbheading{Bibliography, Acronyms and Figures}}

So the combination of \LaTeX~and \BibTeX~already helps us for this separation.
Using the acronym package, we can extend this to acronyms. Looking into the highly
common re-use of figures, we should look into this as well. Let's take the organisation 
of bibliographic information as a template. I store them using \BibTeX and process them 
with the biblatex package (but that is not critical for now). My \BibTeX database is in 
a special folder (we can call it \skbem[code]{references} for the moment) and it uses 
a file structure that helps me to find information. This structure is based on the \BibTeX
and biblatex classification, i.e. inproceedings, article, book, thesis, standard, etc.

Now, I can do the same for figures: put them into a special folder (we can call it
\skbem[code]{figures} for the moment), which contains the source of the figures and the
generated formats I need for my documents (usually \skbacft{org:iso:pdf}, some \skbacft{org:ietf:png}). Now I can reference these 
figures from the repository as well as for other use cases, such as web publishing or 
presentations\footnote{My figures are exclusively in \skbacft{org:w3c:svg} using inkscape (www.inkscape.org).
This has the advantage of a standardised, text-based format with many export options. All my
figures are in a single root folder, structured very similar to the 
document folders created above, but separated from it. This makes re-use of figures very easy.}.

Last not least, the \skbem[code]{acronym} package 
allows for an automatic handling of acronyms, including list of acronyms. 
It is very similar to {\scshape Bib\TeX} in that it will only show the 
acronyms used in a document out of a (potentially large) database.

One might also want to create other specific structures, such as for programming code. Dont'
stop yourself, it's easier to combine things later (if the separation is not effective) than 
to separate things that are hugely integrated into each other. For one of my internal projects,
a parser generation environment based on \skbacft{it:antlr}, I created a special folder for the EBNF specifications
along with railroad diagrams. Now I can use them in my book and my papers.

Now we name the folders for the separated content. This is straight forward, although you might want 
to use different names (don't worry, the skb supports that). We add to the already created repository 
folder the things we need for figures (\skbem[code]{figures}) and combine acronyms and \BibTeX in a 
folder called \skbem[code]{database}, separating the data from all other content~\footnote{Now, the
reason for the database folder and it's structure 
is that the whole \skbacft{vdm:skb} contains more databases, all of which reside here. If you want to simply apply this
to \LaTeX~documents you might want to use a different strutural approach.}.%
\opt{note}{This slide shows the resulting directory structure}%
\opt{text}{
  Now the directory structure looks like this:
  \begin{longtable}{|>{}p{0.945\textwidth} >{}p{1pt}|}
    \hline
    \rowcolor[gray]{.9}
      \skbinput[from=fig]{dirtree/baf}
    & \\
    \hline
  \end{longtable}
  \addtocounter{table}{-1}
}

What did we do so far? We did separate the different parts of our 
documents. The more clinical you are, the better the result will be. But 
please note: separate as much as you should, not as you could. If you 
don't find a reason for separating 'stuff', then don't do it!