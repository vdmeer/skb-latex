\opt{text}{\skbheading{The Long Story}}

Over several years of writing documents (articles, books, reports, standards, research proposals)
ideas and concepts became distributed (actually a euphemism for 'hidden') within many many documents
(in all sorts of formats) located at many many locations (such as local file system, document management 
system, subversion/perforce systems, web servers, email clients). The problems associated to this situation
are manifold:
\begin{skbnotelist}
  \item Ideas/concepts are hidden, often un-accessible and, as I experienced, search tools are of limited help.
  \item The documents are written in all sorts of formats or available only in (usually proprietary) binary
        formats. Ever tried to open a document written in \skbacft{company:MS} WinWord 6.0 with customised document
        template in a newer version of the same programme? You know then what I am talking about.
  \item Reusing the ideas/concepts, once found in a document and managed to open that very document,
        usually involves huge amount of re-formatting. This will produce mistakes. Ever tried to 
        use a {\scshape Bib}\TeX) generated reference list, found in a \skbacft{ISO:PDF} file in a new article?
        I found better ways to spend my nights and weekends (yes, I am married and I have a garden).
  \item Over time, it can become very difficult to distinguish between different versions of a 
        document, concept and/or idea. As it happens in real life, things move on even in computing
        and the related sciences. Documents are written for a specific historic context, which might
        but often will not appear in their abstract (or the name of the folder their are stored in).
  \item The above issues do apply to figures and presentations as much as to the text part of
        documents. Reorganising my documents/figures/presentations I did find way too many duplicates.
        I have used too many graphic software packages in the past 10 years which don't exist 
        anymore, or which do not run on the latest version of my preferred operating system. Some of
        the figures are only available in some sort of low-resolution bitmap, rendering them useless
        even for a non-peer-reviewed article today (the original source got 'lost', in most cases 
        because someone removed the project folder after the project was terminated).
\end{skbnotelist}

\opt{text}{\skbinput[from=rep]{story/long2}}