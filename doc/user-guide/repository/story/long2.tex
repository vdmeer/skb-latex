A solution is to create a unified document repository, then use 
this repository as 'normative source' to create documents for specific 
purposes while leaving the text blocks, headings, figures, presentations, 
references, acronyms and all other reusable 'stuff' in the repository for 
the next document which might (hopefully will) benefit from them. This can 
(did it for me already) safe a lot of time, demands archiving (of 
published documents, thus creating a traceable history), helps to keep 
important information updated (without jeopardising any other work) and 
prevents losing any 'stuff'.

The repository needs a few rules, a (customisable) structure but beside 
that only a bit of effort to be maintained. To give an example: while 
writing the first version of this article (May 11, 2009), I have moved 4 
lecture notes, 2 presentations, 1 book chapter, 1 book (in writing), 1 
textbook (for students, with 4 chapters) and 4 articles from my 'mess' 
into my repository. This involved some re-formatting (plus the occasional 
re-drawing) to bring the original sources into the target formats. At the 
same time I did develop the rules of my repository, the structure and the 
(mostly \LaTeX) code (and re-wrote/structured/ruled most of them a few 
times). I ended up with 1,314 files in 87 folders, which create 9 articles, 2 
books, 1 textbook, 3 lecture notes and this document (note: the number of 
articles increased, because I could re-assemble 'stuff' for new uses, 
spending some five minutes per one new article). I did remove roughly 100 
pages of text (take the classic Spring \skbacft{Springer:LNCS} format and you get the point 
of the number of characters) and some 40 figures (all duplicates). I did 
find way too many errors in the original sources (most of which have been 
created by 're-using' them earlier from even more-original-sources).
