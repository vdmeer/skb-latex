\opt{text}{\skbheading{Creating a Directory Structure}}

The real power (and possibly madness) of the \skbacft{vdm:skb} comes with the separation of
different parts of a document into different directory structures. For the user guide,
we assume the \opt{text}{following} general directory structure%
\opt{note}{shown in this slide}%
.

\opt{text}{
  \begin{longtable}{|>{}p{0.945\textwidth} >{}p{1pt}|}
    \hline
    \rowcolor[gray]{.9}
      \skbinput[from=fig]{dirtree/complete}
    & \\
    \hline
  \end{longtable}
}

To create this structure, go to the directory were you want to put all your 
documents, say \skbem[code]{/doc}. Now create the folders \skbem[code]{database},
\skbem[code]{figures}, \skbem[code]{publish} and \skbem[code]{repository} and the 
respective sub-folders as shown \opt{text}{above.}\opt{note}{in this slide.} Finally, configure the 
\skbacft{vdm:skb} by either editing one of the configuration files or adding the following line
to all of your published documents (and of course change the text \skbem[code]{myfile.tex}
to something that tells you about the location of the configuration change):

\begin{lstlisting}[style=generic,language=TeX]
\skbconfig[root=/doc,
           acr=database/latex,acrfile=acronym,
           bib=database/bibtex,bibfile=bibliograhpy,
           rep=repository,pub=publish,
           fig=figures,sli=slides
          ]{myfile.tex}
\end{lstlisting}

The directory structures for the publish folder and the repository folder reflect different
views to your document base. The publish folder contains documents that are published for a
reason (i.e. articles, books, presentations, white papers, work in progress) while the 
repository folder contains content, most likely structured following a content-specific 
categorisation. The choice of how the directory structure looks like is yours, and of course 
you could also have multiple document directories with completely different structures, for instance
one for computer science publications and one for a gardening book. The \skbacft{vdm:skb} does not set any 
limit, since it can be configured very flexibly to your needs
\opt{text}{(please see \autoref{macro:skbconfig} for more details)}%
.
