\opt{text}{\skbheading{Rebuild the SKB from Source}}

The \skbacft{A3DS:SKB} class and style files as well as the documentation
can be rebuild from its sources very easily. The class and style
files are part of the documented \LaTeX~sources in the file
\skbem[code]{source/skb.dtx} and the \LaTeX~installer (\skbem[code]{source/skb.ins})
provides all necessary instructions. Simply follow the steps shown in the first part of
\opt{text}{the following example}\opt{note}{this slide}. All you have to
do then is to copy the files created to your \LaTeX~distribution.

\opt{text}{
  \lstinputlisting[style=generic,language=TeX,lastline=5]{\skbfileroot{examples/rebuild}}
}

The \skbacft{A3DS:SKB} documentation comes in several different ways. The file \skbem[code]{source/skb.dtx}
contains the documented source while the files in \skbem[code]{doc/user-guide} can be used
to generate the User Guide without source documentation, the \skbacft{A3DS:SKB} presentation (animated
and not animated) and the lecture notes for the presentation.

When processing the file \skbem[code]{source/skb.dtx}, the User Guide will automatically be
included in the generated \skbacft{ISO:PDF} if it is found in either of the directories 
\skbem[code]{source/../doc/user-guide} (when using the \skbacft{A3DS:SKB} original distribution)
or \skbem[code]{source/../doc/latex/skb/user-guide} (when the \skbacft{A3DS:SKB} is already installed with 
your \LaTeX~distribution).

\opt{text}{The following example shows}%
\opt{note}{The rest of this slide and the next slide show}%
how to generate the complete \skbacft{A3DS:SKB} documentation. Please note that the sequence is partially important, for instance the 
file \skbem[code]{ug-slides-noanim} must be processed before the file \skbem[code]{ug-slides-notes}.

\opt{text}{
  \lstinputlisting[style=generic,language=TeX,firstline=7]{\skbfileroot{examples/rebuild}}
}

Please note that the \skbacft{A3DS:SKB} documentation is heavily using the \skbacft{A3DS:SKB} macros, so you will
need to have the style and class files installed before you can rebuild the documentation.