\opt{text}{\skbheading{Applicability and Side Effects}}

The SKB can help you if
\begin{skbnotelist}
  \item You have too many concepts and ideas spread over too many places.
  \item You want to re-organise all of your 'stuff'.
  \item You plan a 'personal' repository or a well-maintained document base for a single purpose.
        It will be difficult to use the \skbacft{A3DS:SKB} for a widely distributed repository that is not
        well-maintained, since there are side effects that might run out of control.
\end{skbnotelist}

The \skbacft{A3DS:SKB} has side effects which might cause problems
\begin{skbnotelist}
  \item Separating contents from structure (see below) means that the contents needs to be 
        almost context-free. Since we can assemble any contents into 'anything' (a book, an article, ...)
        we need to write self-contained text. That can be very difficult and lead to documents that do not give the 
        reader the impression of a consitent and coherent story. However, carefully writing can avoid that problem,
        and I am sure you are carefull writer already \smiley.
  \item Changes in the repository will potentially effect multiple documents. These changes
        are not necessarily intended or wanted. I.e. if an article, a book and a beamer
        presentation access the same source code example, a change here might have a negative
        effect on the beamer presentation. Similarly, if text is changed it might have a negative
        effect on lecture notes or annotated beamer slides.
  \item Versioning the repository (text and figures) is very difficult.
  \item Cross-references need to be used carefully, since a repository file cannot assume that 
        the master document uses the (other) file referenced. The SKB provides no solution for this 
        at the moment (though I have some ideas).
\end{skbnotelist}