\opt{text,note}{
  This \LaTeX~package can help to build a repository for
  long living documents. It focuses on structure and re-use of text, code,
  figures etc. The basic concept is to first separate structure from content
  (i.e. text about a topic from the structure it is presented by) and then
  separating the content from the actual published document, thus enabling
  easy re-use of text blocks in different publications (i.e. text about a
  protocol in a short article about this protocol as well as in a book about
  many protocols); all without constantly copying or changing text. As a side
  effect, using the document classes provided, it hides a lot of
  \LaTeX~from someone who just wants to write articles and books.
}
\begin{skbmodeslide}
\opt{anim}{
  This \LaTeX~package can help to build a \color{blue}\textbf<2>{repository} \color{black} for
  long living documents. It focuses on \color{blue}\textbf<3>{structure and re-use} \color{black} of text, code, \newline
  figures etc. The \color{blue}\textbf<4>{basic concept} \color{black} is to first \color{blue}\textbf<5>{separate} \color{black} structure from content
  (i.e. text about a topic from the structure it is presented by) and then
  separating the content from the actual published document, thus \color{blue}\textbf<6>{enabling
  easy re-use} \color{black} of text blocks in different publications (i.e. text about a
  protocol in a short article about this protocol as well as in a book about
  many protocols); all without constantly copying or changing text. As a side
  effect, using the document classes provided, it hides a lot of
  \LaTeX~from someone who just wants to write articles and books.
}
\opt{noanim}{
  This \LaTeX~package can help to build a \color{blue}\textbf{repository} \color{black} for
  long living documents. It focuses on \color{blue}\textbf{structure and re-use} \color{black} of text, code,
  figures etc. The \color{blue}\textbf{basic concept} \color{black} is to first \color{blue}\textbf{separate} \color{black} structure from content
  (i.e. text about a topic from the structure it is presented by) and then
  separating the content from the actual published document, thus \color{blue}\textbf{enabling
  easy re-use} \color{black} of text blocks in different publications (i.e. text about a
  protocol in a short article about this protocol as well as in a book about
  many protocols); all without constantly copying or changing text. As a side
  effect, using the document classes provided, it hides a lot of
  \LaTeX~from someone who just wants to write articles and books.
}
\end{skbmodeslide}