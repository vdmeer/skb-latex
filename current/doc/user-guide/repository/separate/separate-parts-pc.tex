\opt{text}{\skbheading{Publications and Content}}

Here is were it might get slightly more complicated than in the first few 
steps. And you might see already that the reason for that is separation! 
We didn't finish the separation, we have to go one step further. And that 
means to separate now the contents (with the references and acronyms and 
figures) from the reason to publish a document. This last step of 
separation is more conceptual, being focused on the \textit{why?} and 
\textit{where?} and \textit{how?} we publish, rather than being focused on 
the \textit{what?} we publish. 

So we do publish for many reasons: articles for research, project 
proposals, reports, lecture notes, standard documents, annotated 
presentations, sometimes even books. We publish for a specific purpose, in 
a specific (soon historic) context, using the requested format (and style 
sheets) and a particular structure of our document that fits the purpose. 
That means we organise and structure our content every time according to 
these constrains. Thus we need a new directory structure for that, since 
we will not reuse that as often as our 'stuff' itself. Remember, we use 
the skb macro \cmd{\skbheading} for headings, not the classical \LaTeX~macros 
like \cmd{\section}, so our files effectively do not contain much information 
about their place in the structure, only that they claim one
\footnote{Currently experimental, but soon to be ready, there will be an extension
to the \cmd{\skbheading} macro that allows a little bit more information to be put 
in the repository files. For the moment this is captured in the \cmd{\skbheadingduc} macro.}.
This comes in handy now, since all we have actually to do is to assign a document heading 
level to every repository file we load. Let's create a folder for the published documents and 
call it \skbem[code]{published} with a set of sub-folders that help us to understand the general
context of the publication.%
\opt{note}{My directory structure is shown in this slide}%
\opt{text}{
My directory structure could look like this:
  \begin{longtable}{|>{}p{0.945\textwidth} >{}p{1pt}|}
    \hline
    \rowcolor[gray]{.9}
      \skbinput[from=fig]{dirtree/publish}
    & \\
    \hline
  \end{longtable}
  \addtocounter{table}{-1}
}

\opt{text}{\skbinput[from=rep]{separate/separate-parts-pc2}}