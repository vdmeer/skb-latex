We could, and it usually makes sense to do so, go one step further in that 
separation. This time within the documents in the \skbem[code]{published} 
folder. The reason is the structure of \LaTeX~documents: they do need some 
commands specific to \LaTeX, which we don't necessarily want to mix with the 
commands that input our content (the files from repository).
So it would make sense to have another pair of documents here, one 
containing all \LaTeX~commands needed to create a document, and one 
containing all the commands that include our content. Say we have a few 
articles for state of the art discussions on \textit{naming}, 
\textit{object-models} and \textit{protocols}, we could create
\opt{text}{the following structure}\opt{note}{the structure shown in this slide}
for the \skbem[code]{article} folder%
\opt{text}{:%
  \begin{longtable}{|>{}p{0.945\textwidth} >{}p{1pt}|}
    \hline
    \rowcolor[gray]{.9}
      \skbinput[from=fig]{dirtree/publish-art}
    & \\
    \hline
  \end{longtable}
  \addtocounter{table}{-1}
}

Now everything is structured, thus simple again. If we are looking for content, we go to the 
\skbem[code]{repository} directory and the directory names help us to find what 
we are looking for. If we need a figure, we do the same at the \skbem[code]{figures} directory. 
\skbem[code]{acronyms} and \skbem[code]{bibtex} help with the respective 
other content. If we want a specific published document, we simply check the 
\skbem[code]{published} directory and will have a look into a 
\skbem[code]{tex} sub-directory to see what content is include how.

Good news, the separation is finished! What have we done? We have 
separated the contents from the structure by creating, created a separate directory structure for 
figures, another one for bibliographic data, one for acronyms and finally a 
complete directory structure for published documents. Content and title 
form a pair, include figure, use acronyms and references and are combined 
in the published documents. At this point we can start calling it document repository.