\opt{text}{\skbheading{The Concept: Separate Things}}

You already got the idea that separation is important, reading about published documents and
a normative repository. The basic idea is: think separation -- separate as much as you can, but
not more. I know that this sounds like a strange idea when the goal is a unified repository, but it is 
essential. So we separate several concerns (taking a concept of distributed system design). So if we
want to facilitate re-usability, we have to:
\begin{skbnoteenum}
  \item separate content of a document from its structure and
  \item separate the different parts of a document.
\end{skbnoteenum}

\noindent For the impatient:
\begin{skbnoteenum}
  \item Separating content from structure means to identify small, coherent blocks of information, i.e.
        text describing a certain aspect or an example, and put them separated into the repository folder.
  \item Separating parts of a document means to separate the part that is important for publishing from 
        the part that is important for the content and put them into different places (one in the publish
        folder and the other one in the repository older). It also means to build a separate repository 
        for figures, since they could be used in many different small blocks of information.
\end{skbnoteenum}